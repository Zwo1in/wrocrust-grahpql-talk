\documentclass{beamer}
\usepackage{minted}
\usepackage{forest}

\title{Rust and GraphQL for backend}
\subtitle{Schema creation and basic features}
\author{Maciek Zwoliński}
\date{June 2022}

\begin{document}
  \frame {
    \maketitle
  }
  \note {
    GraphQL is a powerful and strongly typed way to expose your services for the clients. It is more and more often seen in various backends.

    During this talk we will learn what is graphql and how to use it. We will play with it's basic features and try to create our own graphs of data.
  }
  \note {
    Not-really-sure-what developer trying to take Rust for everything. I'm currently working in SatRev, developing ground and flight software for managing nanosatellites.
    Also cats, linux and pizza lover.
  }
  \note {
    GraphQL is a query language for describing the resources you want to get and also all the machinery for executing those queries in your backend.
    It was created by facebook for their internal use and then open sourced in 2015.
    Since then it gained some popularity and was implemented in many popular programming languages.
  }
  \begin{frame}{Graphql Schema}
    \centering
    \begin{forest}
      [Root
        [Query
          [users]
          [userById]
        ]
        [Mutation (optional)
          [addUser]
          [removeUser]
        ]
        [Subscription (optional)
          [userRegistered]
        ]
      ]
    \end{forest}
  \end{frame}
  \begin{frame}[fragile]
    \scriptsize
    \begin{minted}{rust}
        struct User {
            id: usize,
            name: String,
            surname: String,
            age: u32,
            car_id: Optional<usize>,  // references car.id
        }

        enum Brand {
            Fiat,
            Ford,
            Opel,
        }

        struct Car {
            id: usize,
            brand: Brand,
            model: String,
            year: u32,
            owner_id: usize,  // references user.id
        }
    \end{minted}
  \end{frame}
  \note {
    Do you know REST?

    Consider having some resources, let's say users and cars. We could define them as follows:

    We're gonna use Rust for our definitions. We'll also make it look like if we had some relational database under the hood even tho there will be none during this talk :)

    We'd probably want to have one to many relation for the user and his cars but let's just keep it simple at this moment.
  }
  \begin{frame}[fragile]{REST API}{Overfetching}
    \begin{columns}
        \begin{column}{0.48\textwidth}
          \scriptsize
          \begin{minted}[autogobble]{javascript}
              GET /users -> [
                {
                  "id": 0,  // this is unused
                  "name": "Will",
                  "surname": "Smith",
                  "age": 32,
                  "car_id": null  // so is this
                },
                ...
              ]
          \end{minted}
        \end{column}
        \begin{column}{0.48\textwidth}
          Users:
          \begin{itemize}
            \item Will Smith, 32 years
            \item ...
          \end{itemize}
        \end{column}
    \end{columns}
  \end{frame}
  \begin{frame}[fragile]{REST API}{Underfetching}
    \begin{columns}
        \begin{column}{0.48\textwidth}
          \scriptsize
          \begin{minted}[autogobble]{javascript}
              GET /users -> [
                {
                  "name": "Will",
                  "car_id": 1,
                  ...  // unused
                },
                {
                  "name": "Agnieszka",
                  "car_id": 3,
                  ...  // unused
                },
                ...
              ]
              GET /cars/1 -> {
                "brand": "Opel",
                "model": "Astra",
                "year": 1999  // unused
              }
              GET /cars/3 -> {
                "brand": "Ford",
                "model": "Focus",
                "year": 2010  // unused
              }
          \end{minted}
        \end{column}
        \begin{column}{0.48\textwidth}
          Users:
          \begin{itemize}
            \item Will, Opel Astra
            \item Agnieszka, Ford Focus
            \item ...
          \end{itemize}
        \end{column}
    \end{columns}
  \end{frame}
  \begin{frame}[fragile]{GraphQL API}{Declarative fetching}
    \begin{columns}
        \begin{column}{0.2\textwidth}
          \scriptsize
          \begin{minted}[autogobble]{javascript}
              query {
                users {
                  name,
                  surname,
                  age
                }
              }



              query {
                users {
                  name,
                  car {
                    brand,
                    model
                  }
                }
              }

          \end{minted}
        \end{column}
        \begin{column}{0.35\textwidth}
          \scriptsize
          \begin{minted}[autogobble]{javascript}
            "users": [
              {
                "name": "Will",
                "surname": "Smith",
                "age": 32,
              },
              ...
            ]


            "users": [
              {
                "name": "Will",
                "car": {
                  "brand": "Opel"
                  "model": "Astra"
                }
              },
              ...
            ]
          \end{minted}
        \end{column}
        \begin{column}{0.44\textwidth}
          \footnotesize
          Users:
          \begin{itemize}
            \item Will Smith, 32 years
            \item ...
          \end{itemize}
          \vspace{5.5em}
          Users:
          \begin{itemize}
            \item Will, Opel Astra
            \item ...
          \end{itemize}
          \vspace{5em}
        \end{column}
    \end{columns}
  \end{frame}
  \begin{frame}[fragile]{REST API}{Overfetching/Underfetching}
    \begin{minted}{javascript}
      GET /users_with_cars ?

      GET /users?fields=name,surname ?

      GET /users?cars=true&fields=name,surname
          &car_fields=brand,model ?
    \end{minted}
  \end{frame}
  \note {
    Should we make an endpoint users\_with\_cars which will have different model of what it returns? Or maybe cars\_with\_owners would better suit our potential api consumers?

    Or maybe we could add optional query params to control how fetched resources look like?

    But this would be a complete mess when writing api documentation and having many different possible schemas returned from one endpoint
  }
  \frame {
    Server side crates:

    \quad- \underline{async-graphql}

    \quad- juniper

    \vspace{0.5em}

    \small{both integrates well with popular backend crates like actix-web, axum, rocket, warp...}

    \vspace{2em}

    Client side crates:

    \quad- graphql\_client

    \quad- cynic

    \vspace{0.5em}

    \small{you can choose from different designs of strong typing but subscriptions are not quite there yet}
  }
\end{document}
